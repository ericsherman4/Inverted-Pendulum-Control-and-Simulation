\documentclass{article}
\usepackage[utf8]{inputenc}
\usepackage{mathtools, amssymb, amsthm, amsmath} 
\usepackage{indentfirst}
\usepackage{chngcntr}
\usepackage{graphicx}
\graphicspath{./}
\usepackage{url}

% \title{Simulation of an Inverted Pendulum and Controller}
% \author{Eric Sherman}
% \date{April 29 2022}

\begin{document}

\begin{titlepage}
    \begin{center}
        \vspace*{1cm}
            
        \LARGE
        \textbf{Final Project - Simulation of a Inverted Pendulum and Controller}
            
        \vspace{0.5cm}
        \Large
        Computational Modeling and Simulation (ECE 1180)
        
            
        \vspace{0.5cm}
        \large
        Eric Sherman \\
        April 29 2022
        
        \vspace{3cm}
        \includegraphics[width = 12 cm]{titlepage.png}
            
        \vfill
        
    \end{center}
\end{titlepage}


\section{Introduction}

In this project, I explore the dynamics of an inverted pendulums and potential methods used to control it. An inverted pendulum is a rod-cart system. A cart (represented as a rectangular prism with 4 wheels attached) is restricted to just moving along the x-axis by applying a push/pull force on the cart. A rod is a fixed to the top of the cart and is free to swing on a hinge that restricts the motion to just be along the x-y plane. The rod is allowed to swing the full 360 degrees meaning it is allowed to swing (and clip) through the cart.

The goal for the system is to achieve a final state (defined by the user) which almost always calls for balancing the rod in the vertical upright position. This means that a controller had to be implemented because the system by itself is unstable. The rod will always fall over unless there is an external force applied to balance to the cart to position it in a way that prevents the rod from falling. I've wanted to simulate this system for awhile now as it is a classic controls problem and I've seen it many times on social media, literature, and in lecture examples. Due to it's non-liner nature and being a single-input and multi-output system, there's many advanced control techniques required in order to be able to balance the rod in a perfectly vertical position. Several control methods were explored throughout this project including PID and state space control. 

\section{Motivation}

An inverted pendulum is a classic non-linear controls problem used in academia to teach state state representation, reinforcement learning, state space control, and linearization. My motivation behind this project was to scratch the surface of some of these controls and math topics by going through the full simulation of the system and the controller. Because everything about the system is simulated, an additional effort was needed to discretizatize the system and figure out the order in which to update the system state.

As mentioned, this is a classic controls problem frequently used in academia as a teaching example. This model can be used to explore what happens when certain parameters of the system or the controller are changed and the performance of different controllers could be compared. Additionally, as an extension, the effect of disturbances or time delay can be visualized using this system as well. 

\section{Background}
\subsection{System Dynamics}

\begin{center}
    \includegraphics[width = 10cm]{Freebodydiagram.png} \\
    Figure 1. System model of the inverted pendulum. Source: [1]
\end{center}
\vspace{3 mm}

\noindent The cart-rod system is defined as follows:
\begin{enumerate}
    \item A cart is allowed to move latterly (restricted to x-axis) and is assigned a mass, $M$.
    \item The cart has friction in its wheels and this is simply modeled as $F_{friction} = -b\Ddot{x}$ where $b$ is the coefficient of friction.
    \item A rod, attached to a hinge at the top center of the cart, is allowed to swing freely (full 360 degrees of motion). There is not friction in the hinge.
    \item The rod has a mass $m$ and length of $2*l$, where $l$ is the distance to the center of the cart. It has a moment of inertia of $I$ given by $\cfrac{ml^2}{3}$. It is important to remember that $l$ represents half the length of the rod as this is used throughout the calculations.
    \item The carts x-position is represented by $x$. 
    \item The cart is pushed or pulled by a force $F$. 
\end{enumerate}

The equations can be derived using $F=m*a$ and $\tau = I* \alpha$. The dynamics are derived by first summing all the forces in the horizontal direction for the cart and then summing the horizontal forces on the rod. This system is single-input and multi-output so there will be two dynamics equations needed to represent the whole system. The first equation comes from the horizontal forces on the cart and rod as mentioned above and the second equation comes from the forces on the perpendicular axis of the pendulum. The focus of this project was more so on discretization of this system and on the controller, therefore I chose to use the equations provided by the University of Michigan in [1]. The derivation is also detailed in [2] and [3]. From [1], the final results are: 

\begin{equation}
    F = (M + m)\Ddot{x} + b\Dot{x} + ml\Ddot{\theta}\cos{\theta} - ml\Dot{\theta}^2\sin{\theta}
\end{equation}
\begin{equation}
    (I + ml^2)\Ddot{\theta} + mgl\sin{\theta} = -ml\Ddot{x}\cos{\theta}
\end{equation}

These equations are non-linear equations. Because most control methods can be applied to linear systems, it is required that these equations be linearized. Using the small angle theorem and other approximations and linearizing about the vertical equilibrium angle ($\theta = \pi$), we can greatly simplify the equations. These approximations are detailed in verification section. However, this means that in order for the system to be accurate, it must only operate within small angle deviations from $\theta = \pi$. Due to the linearization about  $\theta = \pi$, we define $\theta = \phi + \pi$ where $\phi$ is a small deviation from equilibrium.  After linearization (as detailed in [1]), the final equations become: 

\begin{equation}
    (I+ml^2)*\Ddot{\phi} -  mgl\phi = ml\Ddot{x}
\end{equation}
\begin{equation}
    (M+m)\Ddot{x} + b\Dot{x} - ml\Ddot{\phi} = u
\end{equation}

Note that $F$ has been substituted with $u$ in equation (4) and will represent the input to the system or the force on the cart.

Since this system will be simulated, it is helpful to represent the system in an influence diagram. The elements in the system influence each other in the following ways:

\begin{center}
    \includegraphics[width = 12cm]{influence_diag.png}
    Figure 2. Influence diagrams.
\end{center}

Continuing the derivation of the dynamics of the system, we can take the Laplace transform of these equations (as shown in [1]), yielding:

\begin{center}
    \includegraphics[width = 12cm]{transfer1.png}
    \includegraphics[width = 12cm]{transfer2.png}
    
\end{center}

where $q = (M + m)(I + ml^2) -(ml)^2$. If we take the inverse Laplace transform, we could see the functional relationships between the input and output for the system. As mentioned before, we have one input, the force, and two outputs, the position of the cart and the angle of the rod. The inverse Laplace transform of these functions would allow us to construct a simulation diagram representing the functional relationship between the input and the output, but because these systems are of higher order, completing this calculation would be complex. 


As mentioned above, taking the inverse Laplace transform to get to the time domain representation would be too complex, so simulating the system with the time domain input-output equations wouldn't be feasible. Instead, I use state-space representation, which nicely models the system as a set of input and output matrices in the form of first order differential equations. The continuous-time state space representation is in the following form:

\begin{align}
    \Dot{x} = Ax + Bu \\
    y = Cx + Du
\end{align}

\noindent where A is the state matrix, B is the input matrix, C is the output matrix, D is the feed-forward matrix (in my case, just a zero matrix), x is the state vector, $\Dot{x}$ is the derivative of the x matrix, y is the output matrix, and u is the input matrix. For the inverted pendulum, the matrices are: 

\begin{center}
    \includegraphics[width=12cm]{statespace.png}
\end{center}

The output $y$ is a 2x1 matrix where the first element is the cart's position and the second element is the deviation from the rod's equilibrium position. My simulation uses this state space representation in order to get next state values. 


\subsection{State Space Discretization}

In discrete time, the $\Dot{x}$ matrix is actually the next state matrix at $t + \Delta{t}$ instead of the derivative [8]. Using [7], we can discretize the state space representation:

\begin{align}
    \cfrac{dx}{dt} = \Dot{x} \approx \cfrac{x(k+1)- X(k)}{\Delta{T}}
\end{align}

\noindent where $\Delta{T}$ is the sampling interval. This approximation can be made only if $\Delta{T}$ is small. Simplifying: 

\begin{align}
    \cfrac{x(k+1)- x(k)}{\Delta{T}} & = Ax(k) + Bu(k) \\ 
    x(k+1) & = x(k) + A\Delta T x(k) + B\Delta T u(k) \\
    x(k+1) & = (I + A\Delta T)x(k) + B\Delta T u(k)
\end{align}

\noindent where I is the identity matrix:

\begin{equation}
    I = 
    \begin{bmatrix}
        1&0&0&0 \\
        0&1&0&0 \\
        0&0&1&0 \\
        0&0&0&1
    \end{bmatrix}
\end{equation}

\noindent and $x(k+1)$ contains the next state values. Equation 10 uses the forward Euler method. In my model, I use the backward Euler method changing Equation 10 to:

\begin{align}
    x(k+1) & = (I - A\Delta T)^{-1}x(k) + B\Delta T u(k)
\end{align}

\subsection{State Space Controller}

In this simulation, I used LQR (Linear Quadratic Regulator) as a controller. The use of a PID controller was explored but because this is a multi-output system, I would need to two PID controllers working in tandem. Therefore, LQR was a better option in this scenario.

LQR will produce a matrix of gains which is used in the update function:

\begin{equation}
    u = -Kx
\end{equation}

\noindent where x is the error, giving:

\begin{equation}
    u = -k(measured - setpoint)
\end{equation}

\noindent as detailed in Chapter 8 of [5]. K is in the from:

\begin{equation}
    K = 
    \begin{bmatrix}
        K_x & K_{\Dot{x}} & K_{\phi} & K_{\Dot{\phi}}
    \end{bmatrix}
\end{equation}

Both $measured$ and $setpoint$ are matrices with the same dimensions as the state matrix. In this case, measured is just the state matrix, $x(k+1)$. We fill the setpoint matrix with the desired end states of the system. In my case, I want the final cart position to be 1 meter, the angle deviation of the rod to be 0, the cart velocity to be 0, and the angular velocity of the rod to be 0, although these can be modified in the code. This gives:

\begin{equation}
    setpoint = 
    \begin{bmatrix}
        1 \\
        0 \\
        0 \\
        0 \\
    \end{bmatrix}
\end{equation}

In order for LQR to produce the K matrix, we need the A and B matrix from earlier, and also a Q matrix and R value. R is the cost to control so if R is low, this means it doesn't cost much for us to actuate this system, meaning the controller will actuate it aggressively if needed. With a high R, this means the cost of controlling a system is high, so the controller will tend to be more conservative. Q the matrix contains weights on each of the state values. It is in the form:

\begin{equation}
    Q = 
    \begin{bmatrix}
        W_x & 0 & 0 & 0 \\
        0 & W_{\Dot{x}} & 0 & 0 \\
        0 & 0 & W_\phi & 0 \\
        0 & 0 & 0 & W_{\Dot{\phi}} 
    \end{bmatrix}
\end{equation}

\noindent where $W_x$, $W_{\Dot{x}}$,  $W_\phi$, and $W_{\Dot{\phi}}$ are weight values for each component of the system state. For example, if we want the controller to be really concerned about $W_\phi$, we would give it a value of 10 or 100. An optimal Q matrix I have found is:

\begin{equation}
    Q = 
    \begin{bmatrix}
        10 & 0 & 0 & 0 \\
        0 & 1 & 0 & 0 \\
        0 & 0 & 1000 & 0 \\
        0 & 0 & 0 & 10 
    \end{bmatrix}
\end{equation}

\noindent It is possible to make the controller unstable depending on your Q matrix. I have found having a have a high position and velocity values for either the cart or the rod can result in unstable behavior because these states are directly related to each other.

Lastly, we add limits to the controller output ($u$) to prevent it from producing unreasonable forces on the cart. This limit can be modified in the $sim\_inputs$ class. 

\subsection{Simulation Update}

The update order is as follows:

\begin{enumerate}
    \item Get next state using $x(k+1)= (I + A\Delta T)x(k) + B\Delta T u(k)$
    \item Calculate the controller output using $u(k+1)= -K[x(k+1) - setpoint]$
    \item Calculate the new system output with the new u value, \\ $y(k+1) = Cx(k+1)+Du(k+1) $
    \item Update the simulation visuals and labels.
    \item Set $x(k)$ equal to $x(k+1)$, making the next value now the current state for the next loop iteration.
    \item Return to 1.
\end{enumerate}

\subsection{Selection of Model Parameters}

Most of the model parameters were used from [1]. I chose to lengthen the rod to make it easier to stabilize. Additionally, I chose a very small $\Delta t$ so that the discretization approximation is correct. At a high $\Delta t$ values, the system will become unstable. 

In terms of the controller, I tuned the Q matrix until I was able to get an optimal response. The R value I chose to be fairly low because in a simulation, it doesn't cost anything to actuate the system.

Additionally, I chose to add an initial rod offset and an initial force so that the controller has to immediately kick into action to prevent the rod from turning over. I added two customizable disturbances in the form of a cart disturbance and a rod disturbance later on the simulation. It is important that I chose initial values that prevents the rod from having a large enough deviation that it falls out of the small angle approximation region. If this happens, the system model is no longer accurate and the behavior will be off. I implemented a check to warn the user if the rod falls out of the region of nominal operation.


\section{Verification}

\subsection{Controllability}
We first check the controllability of the system using the methods detailed in [4]. I implemented a function to complete this check at the start of the simulation and part the result. The ctrb() function takes the A and B matrix as an input.

\subsection{Linearization Checks}
Because this is a linearized non-linear system, checks were implemented to ensure that the system operated within a region where the linearization was still accurate. In the top right corner of the simulation, a label showing weather the linearization is still accurate or not is shown.

The following linearizations were used (from [1]):

\begin{align}
    \cos{\theta} = \cos{(\pi + \phi)} &\approx -1 \\
    \sin{\theta} = \sin{(\pi + \phi)} &\approx -\phi \\
    \Dot{\theta}^2 = \Dot{\phi}^2 &\approx 0 \\
\end{align}

\noindent These linearizations are shown below:
\vspace{5mm}

\begin{center}
    \includegraphics[width = 8cm]{cos-linear.png}\\
    Figure 3. Cosine approximation. \\
    \includegraphics[width = 8cm]{sin-linear.png} \\
    Figure 4. Sine approximation. \\
        \vspace{8mm}
    \includegraphics[width = 8cm]{squared-linear.png} \\
    Figure 5. $\Dot{\phi}^2$ approximation. \\
\end{center}

I set 20\% error as the threshold for considering the approximations accurate. The fastest approximation to hit this threshold is  $\Dot{\phi}^2$.

\subsection{Visual Verification}

The simulation behaves as expected and can balance the rod upright, even after several different disturbance profiles such as an impulse disturbance in the rod angle, an impulse disturbance in the cart force, and rectangular pulse disturbance in the cart force (applying the same force over time for x seconds). As long as the initial disturbances aren't too great and the angle deviation for the rod stays close to the equilibrium point, the system model is accurate and the controller behaves as expected.

A comparison of this model and other models can be used as well. It is important to note that most other models use a mass-less rod with a point mass at the end. This changes the dynamics significantly. Under the "Python (GEKKO) Solution" in [11], we can see the response for the pendulum system with an initial force on the cart to the left. I have attempted to reproduce this using the parameters under "Verification" in the Scenarios section. The initial response of both systems is almost identical. However, with my system, the cart adjusts for the rod angle error immediately. Unfortunately, if I have reduce the weight for the rod angle error, I will get bad oscillations as the cart tries to reach the set position. If I increase the weight for the rod angle error, then the cart will follow a different trajectory than in [11]. Because the models used aren't exactly identical, this discrepancy in the trajectory of the rod and cart towards the end of the motion is expected. However, both models properly reach the desired end state.

\section{Scenarios}

Here, I have listed sets of initial values that result in satisfying controller responses. These parameters can be found in $plant\_inputs$ or the $sim\_inputs$ class. An * next to the name means that this scenario has been used for verification. If a parameter is not listed, this means the original simulation values were used. This includes things such as cart mass, rod length, rod mass, etc. In scenarios in which disturbances are a part of, I have included the disturbance parameters. If the they are omitted, the following parameters are assumed to be set: rod\_disturbance\_on = False and cart\_disturbance\_on = False. However, these parameters and related disturbance parameters are not hard requirements can be modified by the user if desired.

\begin{enumerate}
    \item Oscillatory Stable Response
        \begin{itemize}
            \item end\_state = np.array([[0], [0], [0], [0]])
            \item weights = [100, 1, 1000, 1]
            \item cart\_starting\_pos\_x = -1
            \item rod\_starting\_angle = 0
            \item init\_force = -500
            \item max\_force\_limit = 10
            \item min\_force\_limit = -max\_force\_limit
        \end{itemize}
        
    \item Oscillatory Unstable Response
        \begin{itemize}
            \item end\_state = np.array([[0], [0], [0], [0]])
            \item weights = [100, 1, 1000, 1]
            \item cart\_starting\_pos\_x = -1
            \item rod\_starting\_angle = -math.pi/7
            \item init\_force = -500
            \item max\_force\_limit = 10
            \item min\_force\_limit = -max\_force\_limit
        \end{itemize}
        
        \begin{center}
            \includegraphics[width=12cm]{unstable response.png}
            Figure 6. Unstable Oscillation.            
        \end{center}

    
    \item Triple Disturbance 
         \begin{itemize}
            \item end\_state = np.array([[1], [0], [0], [0]])
            \item weights = [100, 1, 1000, 1]
            \item cart\_starting\_pos\_x = -1
            \item rod\_starting\_angle = -math.pi/7
            \item init\_force = 800
            \item max\_force\_limit = 10
            \item min\_force\_limit = -max\_force\_limit
            \item rod\_disturbance\_on = True
            \item rod\_disturbance\_at\_time = 7
            \item rod\_disturbance\_angle = -math.pi/6
            \item cart\_disturbance\_on = True
            \item cart\_disturbance\_at\_time = 10
            \item cart\_disturbance\_force = 9
            \item cart\_disturbance\_over\_num\_steps = 30
        \end{itemize}
    
    \item Unstable Controller (saved by limits)
        \begin{itemize}
            \item end\_state = np.array([[1], [0], [0], [0]])
            \item weights = [1, 1, 1000, 100]
            \item cart\_starting\_pos\_x = -1
            \item rod\_starting\_angle = -math.pi/7
            \item init\_force = 0
            \item max\_force\_limit = 10
            \item min\_force\_limit = -max\_force\_limit
        \end{itemize}
        \begin{center}
            \includegraphics[width=12cm]{unstable controller.png}
            Figure 7. The controller acts as a bang-bang controller. Without limits, the controller output explodes to +$\infty$.
        \end{center}
        
        
    \item Verification *
        \begin{itemize}
            \item end\_state = np.array([[0], [0], [0], [0]])
            \item weights = [30, 1, 200, 10]
            \item cart\_starting\_pos\_x = -1
            \item rod\_starting\_angle = -math.pi/12
            \item init\_force = -300
            \item max\_force\_limit = 10
            \item min\_force\_limit = -max\_force\_limit
        \end{itemize}
        \begin{center}
            \includegraphics[width=12cm]{unstable controller.png}
            Figure 7. The controller acts as a bang-bang controller. Without limits, the controller output explodes to +$\infty$.
        \end{center}

\end{enumerate}


\section{Results}


\section{Evaluation}


\section{Conclusions and Future work}
\subsection{Conclusion}

conlusions shiz ehre

\subsection{Future Work}
In terms of future work, I would like to analyze the effects of introducing different delays in the controller feedback. At a large enought time delay, the system should become unstable. Additionally, I would like to explore using other controllers with this system or even taking a machine learning approach - for example, using reinforcement learning. Lastly it would be interesting to potentially explore a system that is not bound to a 2D plane for both the cart and the rod.

\section{Acknowledgements}
\noindent A big thank you to:
\begin{itemize}
    \item Dr. Zhi-Hong Mao, for helping me discretize state space representation and assisting with debugging the controller.
    \item Thomas Detlefsen for helping me learn state space representation.
    \item Matthew Sivaprakasam for assisting with implementing and debugging a PID controller approach.
    \item Dr. Kara Bocan for providing advice and direction on the project.
\end{itemize}


\section{References}

\noindent $[1]$ Inverted Pendulum, System Modeling: \url{https://ctms.engin.umich.edu/CTMS/index.php?example=InvertedPendulum&section=SystemModeling}
\\
\vspace{1 mm} \\
$[2]$ Inverted pendulum: \url{https://en.wikipedia.org/wiki/Inverted_pendulum}
\\
\vspace{1 mm} \\
$[3]$ LCS - 13 - Pendulum on cart system - mathematical modeling and transfer function: \url{https://www.youtube.com/watch?v=c3z4eo6s0Ek}
\\
\vspace{1 mm} \\
$[4]$ Controllability and Observability: \url{https://www.ece.rutgers.edu/~gajic/psfiles/chap5traCO.pdf} \\
\\
$[5]$ Data Driven Science and Engineering; Machine Learning, Dynamical Systems, and Control: \url{http://databookuw.com/databook.pdf} \\
\\
$[6]$ Modeling and Simulation of Inverted Pendulum: \url{https://www.cantorsparadise.com/modelling-and-simulation-of-inverted-pendulum-5ac423fed8ac}
\\
\vspace{1 mm} \\
$[7]$ Discretization: \url{https://en.wikipedia.org/wiki/Discretization} \\
\\
\vspace{1 mm} \\
$[8]$ State-Space Representation: \url{https://en.wikipedia.org/wiki/State-space_representation} \\
\\
\vspace{1 mm} \\
$[9]$ Simulating and controlling an inverted pendulum: \url{https://fab.cba.mit.edu/classes/864.17/people/copplestone/final_project/index.html} \\
\\
\vspace{1 mm} \\
$[10]$ Linear Quadratic Regulator (LQR) Control for the Inverted Pendulum on a Cart [Control Bootcamp]: \url{https://www.youtube.com/watch?v=1_UobILf3cc&list=PLMrJAkhIeNNR20Mz-VpzgfQs5zrYi085m&index=14} \\
\\
\vspace{1 mm} \\
$[11]$ Inverted Pendulum Optimal Control: \url{https://apmonitor.com/do/index.php/Main/InvertedPendulum} \\
\\
\vspace{1 mm} \\
$[4]$ TITLE: \url{URL} \\

\end{document}
